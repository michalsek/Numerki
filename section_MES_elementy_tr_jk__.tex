\section{MES: elementy trójkątne, prostokątne}

\subsection{elementy prostokątne}

Hermite'a

- wartości funkcji oraz pochodnych mają się zgadzać

- funkcja kształtu na elemencie wzorcowym 

C - obszar wzorcowy 

$C = \{ (\xi,\nu): 0 \leqslant \xi,\nu \leqslant 1 \}$

$\alpha _i, \beta _i$ - wielomiany stopnia $2m - 1$

$\frac{\partial \alpha _i}{\partial \xi _j} |_{\xi = 0} = \sigma_{ij}$

$\frac{\partial \alpha _i}{\partial \xi _j} |_{\xi = 1} = 0$

$\frac{\partial \beta _i}{\partial \xi _j} |_{\xi = 0} = 0$

$\frac{\partial \beta _i}{\partial \xi _j} |_{\xi = 1} = \sigma_{ij}$

$G_{\xi_{ij}}(f) = \frac{\partial^j f}{\partial \xi_{ij} \nu^j} |_{P_k}\ \ \ \ \ \ \ \ i,j = 0,...,m-1\ \ \ \ \ k=1,...,h$

Lagrange'a

$\wedge_{ij} (\xi, \nu) = \alpha _i(\xi) \alpha _j(\nu)$

$V_k \subset c_I^\wedge (\bar{D})$

są wzory... (?:D)

\subsection{elementy trójkątne}

Niech $\bar{D}$ będzie sumą pewnej liczby trójkątów które mają wspólny bok albo wspólny wierzchołek albo soą rozłączne

Trójkąty Lagrange'a (1-go stopnia)

Niech $e_1,...,e_E$ trójąty siatki

$P_1,...P_{mw}$ wierzchołki trójkątów

$V_k$ tworzymy następująco:

1) zadajemy dowolne wartości $u \in V_h$ w węzłach $P_i$

2) zacieśnienie $u \in V_h$ do całego trójkąta jest postaci $ax+by+c$

Tw. $V_h \subset C_I^1(\bar{D})$ kawałkami(?)

zacieśnienie:

$u: \bar{D} \rightarrow \mathbb{R}$

$u|_{ek}$ - to jest zacieśnienie = zawężenie (to samo)

Trójkąty Hermite'a (5-go stopnia)

Niech $P_1,P_2,...$ - wierzchołki trójkątów triangulacji

$l_1, l_2,...$ - boki

$s_1, s_2$ - środki boków

$\eta _k$ - wektor normalny do krawędzi $l_k$

$l_k = \bar{P_iP_j} i > j$

$(l_k,\eta_k) > 0$

Dla wierzchołków definiujemy 6 stopni swobody

$F_{i1}(f) = f(P_i)$

$F_{i2}(f) = \frac{\partial f}{\partial x}(P_i)$

$F_{i3}(f) = \frac{\partial f}{\partial y}(P_i)$

$F_{i4}(f) = \frac{\partial^2 f}{\partial x^2}(P_i)$

$F_{i5}(f) = \frac{\partial^2 f}{\partial x \partial y}(P_i)$

$F_{i6}(f) = \frac{\partial^2 f}{\partial y^2}(P_i)$

Każdemu środkowi boku $S_i$ przyporządkowujemy

$F_k(f) = \frac{\partial f}{\partial \eta_k}(s_i)$

$V_h określamy następująco

1) dla $u \in V_h$ zadajemy wartości $F_{ij}(u), F_k(h), \forall i,j,k$

2) $u$ zawężając do pojedynczego trójkąta jest wielomianem stopnia 5