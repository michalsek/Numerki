\textit{13. Stabilność równań typu hiperbolicznego}

$\left\{\begin{matrix} u_{tt} - u_{xx} = 0 & (x,t) \in D \times [0,T) \\ u = g & (x,t) \in \partial D \times [0,T) \\ u(x,0) = p,\ u_t(x_i,0) = g & x \in D \end{matrix}\right.$

Rodzina schematów różnicowych

$(1)_h\ \ \frac{1}{\tau ^2}(v^{n+1} - 2v^n + v^{n-1}) = L_h (\alpha v^{n+1} + (1 - 2\alpha)v^n + \alpha v^{n-1})$

Metoda jawna:

Metoda może być stabilna, jeśli $\frac{\tau ^2}{h^2} > 1$ wtedy

$v^n = \sum_{s=1}{n+1} u_s(C_{1s} \mu_{1s}^n - C_{2s} \mu_{2s}^n)$

gdzie $\mu_{1s},\ \mu_{2s}$ - pierwiastki równania

$\mu_s^2 - 2(1+ \beta _s)\mu_s + 1 = 0\ \ \ \ s = 1, ..., \mu -1$

$v^0,v^1$ - wznaczane z warunków początkowych

\[ (\left | C_{1s} \right | + \left | C_{2s} \right |)^2 \leqslant A(\beta_s^2 + \varphi_s^2) \leqslant B(\left | v^0 \right |^2 + \left | v^1 \right | \]

$\left \| v^n \right \|^2 \leqslant C(\left | v^0 \right | + \left | v^1 \right |) \Rightarrow$ otrzymaliśmyt ciągłą zależność od warunków początkowych zatem mamy stabilność.