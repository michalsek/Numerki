\section{Aproksymacja dla równania przewodnictwa ciepła (schemat jawny, niejawny, Cranka-Nicholsona, rząd metody)}

\textbf{1. Schemat Jawny}

$L_u = u_t - u_{xx}$

$L = \frac{\partial}{\partial t} - \frac{\partial^2}{\partial x^2}\ \ \ \ \ \ \ \ \ (u_t - \bigtriangleup u)$

$D = (0,l) \subset \mathbb{R}$

$Q = D \times (0,T)$

$\bar{Q_{n\tau}} = \bar{D_h} \times \{ n\tau : n = 0, 1, ..., N\ \ \ N\tau = T\}$

$v = \bar{Q_\tau} \rightarrow \mathbb{R}$

Oznaczamy: $V^n_i = u(x_i,t^n)\ \ \ \ (x_t,t^n) \in Q_{n\tau}$

$(L_{h\tau}\ v)(x_i, t^n) = \frac{1}{\tau} (v^{n+1}_i - v^n_i) - \frac{1}{h^2} (v^n_{i+1} - 2v^n_i + v^n_{i-1})$

$(L_{h\tau}\ v)(x, t^n) = \frac{1}{\tau} (v^{n+1}_x - v^n_x - (L_h\ v)(x, t^n)$

$o(\tau, h^2)$ - błąd aproksymacji

wniosek

Mogę wyznaczyć $v^{n+1}_i$, jeśli znam wartości $v^{n}_{i-1}$, $v^{n}_{i}$, $v^{n}_{i+1}$

$(L_{h\tau}\ v)(x, t^n) = 0$

\textbf{2. Schemat niejawny, Crancha-Nicholsona}

Rodzina schematów niejawnych

$(L_{h\tau}\ v^n) = \frac{1}{2} (v^{n-1} - v^{n}) - L_h (\lambda v^{n+1} + (1-\lambda)v^{n})\ \ \ \ 0 \geqslant \lambda \geqslant 1$

$(L_{h\tau}\ v^n) = 0$ - układ równań wektorowych $(v^n_1,\ ...,\ v^n_n)$

dla $\lambda = \frac{1}{2}$ jest to schemat Cranka-Nicholson'a

