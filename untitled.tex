\section{Przykłady zagadnień fizyki i techniki opisywanych prez równania różniczowe cząstkowe}


\textit{- Równania liniowe}


\textit{Równanie Laplace}


Równanie to wyraża następującą własność pola potencjalnego: dywergencja (rozbieżność) wektorowego pola potencjalnego (czyli gradientu potencjału), pod nieobecność źródła jest równa zeru. Opisuje ono zatem wiele procesów zachodzących w przyrodzie, np. potencjał grawitacyjny poza punktami źródeł pola (czyli bez punktów materialnych), potencjał prędkości cieczy przy braku źródeł. Równanie Laplace’a jest szczególnym przypadkiem równania Poissona, wyrażającego analogiczny związek w przypadku istnienia źródeł pola.


$ \bigtriangleup u = \sum_{i=1}^{n} u_{x_ix_i} =0$

$ u_{x_i} = \frac{du}{dx_i}$

$ u_{x_i x_i} = \frac{d^2u}{dx_i^2}$

\textit{Równanie Poisona}



Równanie Poissona opisuje wiele procesów zachodzących w przyrodzie, np. rozkład pola prędkości cieczy wypływającej ze źródła, potencjał pola grawitacyjnego w obecności źródeł, potencjał pola elekrostatycznego w obecności ładunków, temperaturę wewnątrz ciała przy stałym dopływie ciepła.

$ \bigtriangleup u =f$