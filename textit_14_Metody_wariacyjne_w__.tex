\textit{14. Metody wariacyjne w zagadnieniach brzegowych.}

$H$ - przestrzeń Hilberta

$(\ ,\cdot)$ - iloczyn skalarny

$\left \|  \right \|$ - norma

$H_A \subset H, H_A$ - gęsta w $H (\bar{H_A} = H)$

$A: H_A -> H$ odwzorowanie liniowe i ciągłe.

Definicja
$A$ - dodatni $\Leftrightarrow\ \exists K > 0 \forall y \in H_A (Ay,y) \leqslant K \left \| y \right \| ^ 2$