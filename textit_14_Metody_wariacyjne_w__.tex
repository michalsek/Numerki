\textit{14. Metody wariacyjne w zagadnieniach brzegowych.}

$H$ - przestrzeń Hilberta (przestrzeń liniowa nad ciałem liczb rzeczywistych lub zespolonych z abstrakcyjnym iloczynem skalarnym)

$(\ ,\cdot)$ - iloczyn skalarny

$\left \|  \right \|$ - norma

$H_A \subset H, H_A$ - gęsta w $H (\bar{H_A} = H)$

$A: H_A -> H$ odwzorowanie liniowe i ciągłe.

Definicja

$A$ - dodatni $\Leftrightarrow\ \forall y in H_A (Ay,y) \leqslant 0$    dla $y = 0, (Ay,\ y) = 0$

$A$ - określny dodatnio $\Leftrightarrow\ \exists K > 0\ \forall y \in H_A (Ay,y) \leqslant K \left \| y \right \| ^ 2$

$A$ - symetryczny $\Leftrightarrow\ \forall y,z \in H_A\ \ (Ay,z) = (y,Az)$