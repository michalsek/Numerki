\section{7. Twierdzenie Laxa-Filippowa o zbieżności}

(1) $\left\{\begin{matrix} Lu = f & x \in D \\ B_iu = g_i & i = 1,...,n\ \ \ x \in \Gamma _i \end{matrix}\right.$

$(1)_h$ $\left\{\begin{matrix} L_hv = f_h & x \in D_h \\ B_{ih} \tau = g_{ih} & x \in \Gamma _{ih} \end{matrix}\right.$

gdzie:

$L$,$B_i$ - operatory różnicowe (jak w konstrukcji siatki)

Twierdzenie o Stabilności:

Liniowe zadanie przybliżone $(1)_h$ jest stabilne (poprawnie postawione), jeśli:

\[ \exists h_0 > 0 \ \ \ \forall h \leqslant h_0 , h \in \omega \ \ \ \forall f_h \in F_h,\ \  g_{k,h} \in \Phi _{k,h}\ \ \ k = 1,...,s\]

zachodzi:

1. istnieje jednoznacznie wyznaczone rozwiązanie $u_h in U_h$ spełniające $(1)_h$
2. rozwiązanie to spełnia nierówność:

\[ \left \| u_h \right  \|_{U_h} \leqslant C(\left \| f_h \right \|_{F_h} + \sum_{k=1}^S \left \| g_{k,h} \right \|_{\Phi _{k,h}}) \]

gdzie C - to dodatnia stała niezależna od $H$ (ani ofc od $f_h, g_{k,h}$).

Twierdzenie Laxa-Fillipowa

Jeśli liniowe zadanie przybliżone $(1)_h$ jest stabilne oraz aproksymuje zadanie $(1)$, którego rozwiązaniem jest u, wtedy zadanie przybliżone jest zbieżne oraz

\[ \left \| r_h^U u - u_h \right \|_{U_h} \leqslant C(\sum_{k=0}^{s} e_{k,h}) \]

Tutaj $u_h$ to rozwiązanie zadania $(1)_h$