\section{Aproksymacja operatorów różniczkowych (przypadek jednowymiarowy: ${u}', {{u}'}'$; przypadek dwuwymiarowy: $\bigtriangleup u$)}

Aproksymacje przeprowadzamy za pomoc wzoru Taylora. 

$f'$ - aproksymacja

$f(x+h) = f(x) + hf'(x) + R$

$f'(x) = \frac{f(x + h) - f(x)}{h}$

$f''$ - aproksymacja

$f(x+h) = f(x) + hf'(x) + \frac{h^2}{2}f''(x) + R$

$f(x-h) = f(x) - hf'(x) + \frac{h^2}{2}f''(x) + R$

$f(x-h) + f(x+h) = 2f(x) + h^2 f''(x)$

$f''(x) = \frac{f(x-h) + f(x+h) -2f(x)}{h^2} $

$\bigtriangleup f$ aproksymacja (Dwa wymiary)
Korzystajac z aproksymacji f''(x)

$\frac{d^2F}{dx^2} = \frac{F(x-h_1 , y) + f(x+h_1,y) -2f(x,y)}{h_1^2} $

$\frac{d^2F}{dy^2} = \frac{f(x , y-h_2) + f(x, y+h_2) -2f(x,y)}{h_2^2} $

$\bigtriangleup F = \frac{F(x-h_1 , y) + f(x+h_1,y) -2f(x,y)}{h_1^2} + \frac{f(x , y-h_2) + f(x, y+h_2) -2f(x,y)}{h_2^2}$