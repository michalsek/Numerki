\textit{3. Zadania poprawnie postawione. Przykład Hadamarda}

a) Zadanie jest poprawnie postawione jeśli:

  - przy określonych warunkach granicznych istnieje rozwiązanie tego zadania
  
  - rozwiązanie to jest jednoznaczne

  - rozwiązanie to zależy w sposób ciągły od zadanych warunków granicznych

b) Przykład Hadamarda

$\left\{\begin{matrix} u_{xx} + u_{tt} = 0 \\ u(x,0) = 0 \end{matrix}\right.$

$\left\{\begin{matrix} u_{xx} + u_{tt} = 0 \\ u_n(x,0) = e^{-\sqrt{n}}\ dla\ n = 1, ..., n \end{matrix}\right.$

Rozwiązanie: $u_n(x,t) = e^{-\sqrt{n}}e^{nt}cos\ nx\  n = 1, 2, ...$

$u_n(x,0) = e^{-\sqrt{n}}cos\ nx$

$u_{xx} = -e^{-\sqrt{n}}e^{nt}n^2cos\ nx$

$u_{tt} = e^{-\sqrt{n}}cos\ nx n^2 e^{nt}$

$u_{tt} + u_{xx} = 0$

