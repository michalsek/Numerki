\section{Postulaty konstrukcji przestrzeni typu elementów skończonych.}

\subsection{Postulaty Hermita}

- wartości funkcji oraz pochodnych mają się zgadzać

- funkcja kształtu na elemencie wzorcowym 

C - obszar wzorcowy 

$C = \{ (\xi,\nu): 0 \leqslant \xi,\nu \leqslant 1 }$

$\alpha _i, \beta _i$ - wielomiany stopnia $2m - 1$

$\frac{\partial \alpha _i}{\partial \xi _j} |_{\xi = 0} = \sigma_{ij}$

$\frac{\partial \alpha _i}{\partial \xi _j} |_{\xi = 1} = 0$

$\frac{\partial \beta _i}{\partial \xi _j} |_{\xi = 0} = 0$

$\frac{\partial \beta _i}{\partial \xi _j} |_{\xi = 1} = \sigma_{ij}$

$G_{\xi_{ij}}(f) = \frac{\partial^j f}{\partial \xi_{ij} \nu^j} |_{P_k}\ \ \ \ \ \ \ \ i,j = 0,...,m-1\ \ \ \ \ k=1,...,h$

Lemat: Zadanie interpolacji znaleźć wielomian $p(\xi,\nu)$ stopnia $2m-1$ po $\xi$ przy zadanych dowolnych wartościach $G_{\xi_{ij}}(p)$ ma tylko jedno rozwiązanie.

Postulaty Lagrange'a

$\wedge_{ij} (\xi, \nu) = \alpha _i(\xi) \alpha _j(\nu)$

$V_k \subset c_I^\wedge (\bar{D})$

są wzory... (?:D)
