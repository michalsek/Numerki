\section{6. Metody różnicowe: błąd zbieżności, błąd aproksymacji zadania.}

Definicja

Zadanie przybliżone $(1)_h$ jest zbiezne, jesli

\[ \lim_{h \rightarrow 0} \left \| r_h(u) - u_h \right \|_{U_h} = 0 \]

gdzie

$\left \| r_h(u) - u_h \right \|_{U_h}$ - błąd zbieżności

Definicja

błąd aproksymacji (równania i warunków brzegowych)

\[ \bar{X_{0,h}} (u) = \left \| L_h r_h u - f_h \right \| _{F_u} = \left \| L_hr_hu - rh(Lu) \right \| _{F_h} \]
\[ \bar{X_{i,h}} (u) = \left \| B_{i,h} r_h u - g_{i,h} \right \|_{G_{i,h}} = \left \| B_{i,h} r_h u - r_h (B_{i,h} u) \right \| _{G_{i,h}} \]

Błąd aproksymacji zadania

\[ \bar{X_h} (u) = max \{ \bar{X_{0,h}}, \bar{X_{1,h}}, ..., \bar{X_{n,h}} \} \]

aproksymacja jest rzędu $k$ jesli $\bar{X_h} (u) = O(h^k)$ $\ \ \ \ \ \ \ \ O$ - notacja duże O [$ f(x) \in O(g(x)) \Leftrightarrow \frac{f(x)}{g(x)} \rightarrow C (x \rightarrow 0)$]



