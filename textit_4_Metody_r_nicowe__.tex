\textit{4. Metody różnicowe rozwiązywania równań różniczkowych cząstkowych (konstrukcja siatek, przestrzenie funkcji siatkowych, aproksymacja operatorów różniczkowych)}

\textbf{a\) konstrukcja siatek}

$R^{2}_{h} = \{ x = (i_{1}h_{1}, i_{2}h_{2})\ h_{1},h_{2} > 0;\ i_{1},i_{2} \in \mathbb{Z}\}$

siatka $\bar{D_{h}} = D \cap R^{2}_{h}$

$D_{h} = \{ x \in \bar{D_{h}} N_h(x) \subset \bar{D_{h}}\}$

$N_h(x)$ - otoczenie siatkowe

$x$ - punkt wewnętrzny siatki

$\Gamma_h = \bar{D_h} \setminus D_h$

$\omega = \{h, \frac{1}{2}h, \frac{1}{4}h, ...\}\ h \in \omega\ 0 \in \bar{\omega}$

Def.

Rodzinę $\{D_h\}_{h \rightarrow \inf}$ nazywamy gęstą w $\bar{D}$, jeśli $\cup D_h$ jest gęsty w $\bar{D}$