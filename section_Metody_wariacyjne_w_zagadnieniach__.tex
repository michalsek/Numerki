\section{Metody wariacyjne w zagadnieniach brzegowych.}

Rozważmy równanie poisonna

(1) $-((a\tilde{u_x})_x + (c\tilde{u_y})_y) = g\ \ \ \ \ \ \ \ \ (x,y) \in D \subset \mathbb{R}^2$ - obszar ograniczony z regularnym brzegiem

(2) $\tilde{u} = g_1\ \ \ \ \ \ \ \ \ (x,y) \in \Gamma = \partial D$

$\ $

$H$ - przestrzeń Hilberta (przestrzeń liniowa nad ciałem liczb rzeczywistych lub zespolonych z abstrakcyjnym iloczynem skalarnym)

$(\ ,\cdot)$ - iloczyn skalarny

$\left \|  \right \|$ - norma

$H_A \subset H, H_A$ - gęsta w $H (\bar{H_A} = H)$

$A: H_A -> H$ odwzorowanie liniowe i ciągłe.

Definicja

$A$ - dodatni $\Leftrightarrow\ \forall y \in H_A (Ay,y) \leqslant 0$    dla $y = 0, (Ay,\ y) = 0$

$A$ - określny dodatnio $\Leftrightarrow\ \exists K > 0\ \forall y \in H_A (Ay,y) \leqslant K \left \| y \right \| ^ 2$

$A$ - symetryczny $\Leftrightarrow\ \forall y,z \in H_A\ \ (Ay,z) = (y,Az)$

Problem: $f \subset H$

    rozwiązać równanie (1) $Ay=f, y \in H_A$
    
Twierdzenie 1:

$A$ - dodatni $\Rightarrow$ (1) ma conajwyżej 1 rozwiązanie

Twierdzenie 2 "$\Rightarrow$":

$A$ - dodatni, symetryczny, $y_0$ - rozwiązanie (1) $\Rightarrow$ $y_0$ minimalizuje funkcjonał $I(y) = (Ay,y) - z(f,y)$

Twierdzenie 3 "$\Leftarrow$":

$A$ - dodatni, symetryczny, $\bar{y} \in H_A$ minimalizuje $I$ $\Rightarrow$ $\bar{y}$ spełnia (1)


Niech $\bar{z}$ - rozwiązanie (2) min ${ I(y):\ y \in H_A}$

Oznaczmy: $\mu = I(\bar{z})$

Zakładamy $\exists \{z_k\} \subset H_A$ taki że $\lim_{k \to \infty} I(z_k) = \mu$

$\{z_k\}$ - nazywamy ciągiem minimalizującym $I$

Twierdzenie 4:

$A$ - symetryczny, dodatnio określony

Wtedy dowolny ciąg minimalizujący $\{z_k\}$ jest zbieżny do rozwiązania problemu $\bar{z}$