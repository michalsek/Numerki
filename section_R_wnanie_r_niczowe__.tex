\section{Równanie różniczowe cząstkowe; rząd równania różniczkowego; warunki graniczne}
Równaniem różniczkowym cz
stkowym nazywamy równanie funkcyjne w postaci

$F(x ,u(x) , D^\alpha u) = f(x)$

$0<|\alpha| < m$

$u:D \subset R^n -> R^n$

$\alpha$ multiindex (wektor)

$|\alpha| = \sum \aplha_i$

$D^\alpha u = \frac{d^{|\alpha|} u}{dx_1^{\alpha_1} + dx_n^{\alpha_n}}$