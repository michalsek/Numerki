\section{Równanie różniczowe cząstkowe; rząd równania różniczkowego; warunki graniczne}
Równaniem różniczkowym cz
stkowym nazywamy równanie funkcyjne w postaci

$F(x ,u(x) , D^\alpha u) = f(x)$

$0<|\alpha| < m$

$u:D \subset R^n -> R^n$

$\alpha$ multiindex (wektor)

$|\alpha| = \sum \alpha_i$

$D^\alpha u = \frac{d^{|\alpha|} u}{dx_1^{\alpha_1} + dx_n^{\alpha_n}}$


Rzędem równania różniczkowego cząstkowego nazywamy najwyższy rząd pochodnej funkcji niewiadomej występującej w danym równaniu. 

Warunki brzegowe:

$ B_j(x,u,d^\alpha u) = g_j(x) j=1 .. s $
$ 0 < |\alpha| < p_j$
$x \in dD$