\textit{15. Metoda Ritza (metoda wyznaczania ciągu minimalizującego)}

Założenia jak w twierdzeniu 4 (porzedniu punkt)

Zakładamy, że $H_A$ - ośrodkowa, tzn.

definicja - przestrzeń osrodkowa

$\exists \{z_k\} \subset H_A$ takie że:

i) $\forall n\ z_1, ..., z_n$ są liniowo niezależne

ii) $forall \varepsilon > 0\ \ \forall y \in H_A \ \ \exists m\ \ \alpha _1, ..., \alpha _m\ \ : \ \ A(y - \sum_{j=1}^{m} \alpha _j z_j) < \varepsilon$ (???)

$\ $

$\ $

Sposób postępowania

1. Ustalam n i oznaczam $u_n = \sum_{k=1}^{n} \beta _k z_k \in SPAN\{z_1, ..., z_n\}$

\[ I(u_n) = (A\sum_{k=1}^{n} \beta _k z_k \sum_{k=1}^{n} \beta _k z_k) - 2(f, \sum_{k=1}^{n} \beta _k z_k) = \sum_{k=1}^{n}\sum_{j=1}^{m} (Az,z_j) \beta _k \beta _j - 2\sum_{k=1}^{n} (f, z_k) \beta _k = F(\beta _1, ..., \beta _n)\]

2. Szukam minimum funkcji $F$: $min\{F(\beta)\ \ \beta \in \mathbb{R}\}$

(3) $\frac{\partial F}{\partial \beta _k} = 0$ dla $k = 1,...,n$

(3') $2 \sum_{j=1}^{m} (Az_k, z_j) \beta _j - 2(f, z_k) = 0$ dla $k = 1,...,m$

$ (3) \Leftrightarrow (3') $

(3') - układ równań liniowych o niewiadomych $\beta _i\ i = 1,...,n$ io macierzy współczynników (Gramma) $C = \{ Az_k, z_j \}_{k,1 = 1,...,n}$

$C_{kj} = (Az_k, z_j)$ - macierz dodatnio określona, nieosobliwa (bo $z_1,...,z_n$ - liniowo niezależne)

$A$ - dodatnio określony i symetryczny $\Rightarrow$ $(Az,v) =: (z|v)$ - iloczyn skalarny (równoważny z $(z,v)$)

(3') - posiada rozwiązanie $gamma$

Oznaczmy $W_n = \sum_{k=1}^{n} \gamma _k z_k$        $\{\gamma _k\} - rozwiązanie (3')

Twierdzenie:

Założenia jak w Twierdzeniu 4, $\bar{z}$ - rozwiązanie (2) $ min\{ I(y) y \in H_A \}$

$\{W_n\}$ - ciąg $(I(W_n) = min\{I(y): y \in SPAN\{z_1,...,zn\}\}$

Wtedy ${W_n}$ jest ciągiem minimalizującym dla $I$

