\textit{15. Metoda Ritza (metoda wyznaczania ciągu minimalizującego)}

Założenia jak w twierdzeniu 4 (porzedniu punkt)

Zakładamy, że $H_A$ - ośrodkowa, tzn.

definicja - przestrzeń osrodkowa

$\exists \{z_k\} \subset H_A$ takie że:

i) $\forall n\ z_1, ..., z_n$ są liniowo niezależne

ii) $forall \varepsilon > 0\ \ \forall y \in H_A \ \ \exists m\ \ \alpha _1, ..., \alpha _m\ \ : \ \ A(y - \sum_{j=1}^{m} \alpha _j z_j) < \varepsilon$ (???)

$\ $

$\ $

Sposób postępowania

1. Ustalam n i oznaczam $u_n = \sum_{k=1}^{n} \beta _k z_k \in SPAN\{z_1, ..., z_n\}$

\[ I(u_n) = (A\sum_{k=1}^{n} \beta _k z_k \sum_{k=1}^{n} \beta _k z_k) - 2(f, \sum_{k=1}^{n} \beta _k z_k) = \sum_{k=1}^{n}\sum_{j=1}^{m} (Az,z_j) \beta _k \beta _j - 2\sum_{k=1}^{n} (f, z_k) \beta _k = F(\beta _1, ..., \beta _n)\]