\section{3. Zadania poprawnie postawione. Przykład Hadamarda}

a) Zadanie jest poprawnie postawione jeśli:

  - przy określonych warunkach granicznych istnieje rozwiązanie tego zadania
  
  - rozwiązanie to jest jednoznaczne

  - rozwiązanie to zależy w sposób ciągły od zadanych warunków granicznych

b) Przykład Hadamarda
Rozwiązaniem zagadnienia Cauchy'ego dla równania Laplace'a

$\left\{\begin{matrix} u_{xx} + u_{tt} = 0 \\ u(x,0) = 0 \end{matrix}\right.$

jest $u(x,t) \equiv 0,\ x \in \mathbb{R},\ t \geqslant 0$.

Rozwiązaniem ciągu zagadnień:(nie ważne??)

$\left\{\begin{matrix} u_{xx} + u_{tt} = 0 \\ u_n(x,0) = e^{-\sqrt{n}}\ dla\ n = 1, ..., n \end{matrix}\right.$

jest

$u_n(x,t) = e^{-\sqrt{n}}e^{nt}cos\ nx\  n = 1, 2, ...$


